\documentclass[12pt]{article}
\usepackage{pictex,amsmath,amssymb,verbatim,amsbsy,amsfonts,amsthm}
\usepackage{graphics,graphicx}
\usepackage{cases}

\setlength{\voffset}{-0.25in}
\setlength{\headsep}{+0.5in}
\setlength{\parskip}{1em}
\setlength{\parindent}{0em}

\usepackage{fullpage}
\usepackage{fancyhdr}
\usepackage{multicol,multirow}

\def\vu{\mathbf{u}}
\def\vv{\mathbf{v}}
\def\vw{\mathbf{w}}
\def\vb{\mathbf{b}}
\def\vs{\mathbf{s}}

\usepackage{xcolor}
\usepackage{titlesec}
\usepackage{mdframed}
\usepackage[utf8]{vietnam}

\newmdenv[linecolor=blue,skipabove=\topsep,skipbelow=\topsep,leftmargin=5pt,rightmargin=-5pt,innerleftmargin=5pt,innerrightmargin=5pt]{mybox}
\begin{document}
\begin{center}
\textbf{ĐẠI SỐ TUYẾN TÍNH}\\
(Linear Algebra)
\end{center}
\section{Matrix Algebra}
\subsection{Definition}
- A $m \times n$ matrix is a table of numbers that contains in m rows and n columns
\begin{center}
$\begin{pmatrix}
	a_{11} & a_{12} & a_{13} & a_{1n} \\
	a_{21} & a_{22} & a_{23} & a_{2n} \\
	a_{n1} & a_{n2} & a_{n3} & a_{nn}
\end{pmatrix}$
= $(a_{ij})_{m \times n}$
\end{center}
\subsection{Definition}
- If $a_{ij}$ = 0, $\forall i,j$, then A is called a zero matrix and is denoted by 0
\subsection{Definition}
Let A = $(a_{ij})_{m \times n}$:\\
- A matrix A that is formed by columns is called a transpose of A\\
Ex:
$\begin{pmatrix}
	2 & -1 & 3 \\
	4 &  7 & 8
\end{pmatrix}$
=
$\begin{pmatrix}
	 2 & 4 \\
	-1 & 7 \\
	 3 & 8
\end{pmatrix}$
\subsection{Definition}
- If the number of rows and the number of columns equal, then the matrix is called a square matric of order \\
- Let A = $(a_{ij})_{m \times n}$ be a square matrix of order n
\begin{itemize}
	\item All the elements $a_{11}, a_{22}, \ldots, a_{nn}$ is diagonal of A
	\item A sum of $a_{11}, a_{22}, \ldots, a_{nn}$ is called a trace of A \\
	A = 
	$\begin{pmatrix}
		\textbf{1} & 1 & 3 \\
		4 & \textbf{2} & 7 \\
		2 & 5 & \textbf{6}
	\end{pmatrix}$
	\bigbreak
	Trace(A) = 1 + 2 + 6 = 9.
	\item If $a_{ij}$ = 0,$\forall i \not = j$, then A is called a diagonal matrix \\
	D = 
	$\begin{pmatrix}
		4 & 0 & 0 \\
		0 & 2 & 0 \\
		0 & 0 & -1
	\end{pmatrix}$
	is a diagonal matrix
	\bigbreak
	\item
	 $$ 
	 \mbox{If } a_{ij} =
	\begin{cases}
    0, if \mbox{ i} \not = j \\
    1, if \mbox{ i} =j  
	\end{cases}
	$$
    then A is called an identity matrix
\end{itemize}
\section{Matrix Operation}
\subsection{Equality}
A = B $\leftrightarrow$
$$
\begin{cases}
\mbox{Same size} \\
\mbox{Corresponding elements are equal} 
\end{cases}
$$
Ex:
A = 
$\begin{pmatrix}
2 & -1 & 3 \\
4 & 7 & 8
\end{pmatrix}$
\bigbreak
B = 
$\begin{pmatrix}
2 & x & 3 \\
4 & 7 & 8 
\end{pmatrix}$
\bigbreak
A = B when 
$$
\begin{cases}
x = -1 \\
y = 4
\end{cases}
$$
\subsection{Sum:}
$A = (a_{ij})_{m \times n}$\\
$B = (b_{ij})_{m \times n}$ \\
$\rightarrow A + B = (a_{ij} + b_{ij})_{m \times n}$ \\
Ex:\\
A = 
$\begin{pmatrix}
2 & -1 \\
3 & 4
\end{pmatrix}$
\bigbreak
B = 
$\begin{pmatrix}
1 & -3 \\
0 & 7
\end{pmatrix}$
\bigbreak
$\rightarrow{A + B =}$ 
$\begin{pmatrix}
3 & -4 \\
3 & 11
\end{pmatrix}$
\subsection{Multiplication by a number}
$A = (a_{ij})_{m \times n}$ \\
$\rightarrow{ \alpha .A = (\alpha a_{ij})_{m \times n}}$ \\
Ex:\\
A = 
$\begin{pmatrix}
1 & 2 & 1 \\
3 & 5 & 4
\end{pmatrix}$
\bigbreak
$\rightarrow{3A =}$
$\begin{pmatrix}
3 & 6 & 3 \\
8 & 5 & 12 
\end{pmatrix}$
\bigbreak
\subsection{Matrix Multiplication}
$A = (a_{ij})_{m \times n}$ ; $B = (b_{ij})_{n \times p}$ \\
$A \times B = C = (c_{ij})_{m \times p}$ \\
$c_{ij} = (i^{th} \mbox{ rows of A} ) \times (j^{th} \mbox{ columns of B})$ \\
$c_{ij} = (a_{i1} \times a_{i2} \times a_{i3} \ldots \times a_{in})$
$\begin{pmatrix}
b_{1j} \\
b_{2j} \\
\ldots \\
b_{nj}
\end{pmatrix}$
\bigbreak 
$c_{ij} = a_{i1} b_{1j} + a_{ì2} b_{2j} + \ldots + a_{in} b_{nj}$ \\
$c_{ij} = \displaystyle \sum_{n}^{k =1} a_{ik} b_{kj}$ \\
Ex: \\
A =
$\begin{pmatrix}
1 & -1 & 3 \\
2 & 4 & -2 
\end{pmatrix}_{2 \times 3}$
;
B = 
$\begin{pmatrix}
2 & 1 & -1 \\
5 & 2 & 6 \\
0 & 1 & -1
\end{pmatrix}_{3 \times 3}$
\bigbreak
$\rightarrow{A \times B = }$
$\begin{pmatrix}
c_{11} & c_{12} & c_{13} \\
c_{21} & c_{22} & c_{23} 
\end{pmatrix}_{2 \times 3}$
\end{document}